% Created 2021-09-04 Sat 23:27
% Intended LaTeX compiler: pdflatex
\documentclass[11pt]{article}
\usepackage[utf8]{inputenc}
\usepackage[T1]{fontenc}
\usepackage{graphicx}
\usepackage{grffile}
\usepackage{longtable}
\usepackage{wrapfig}
\usepackage{rotating}
\usepackage[normalem]{ulem}
\usepackage{amsmath}
\usepackage{textcomp}
\usepackage{amssymb}
\usepackage{capt-of}
\usepackage{hyperref}
\usepackage[russian, english]{babel}
\usepackage[T2A]{fontenc}
\author{Starovoytov Alexandr}
\date{\textit{<2021-09-04 Sat 13:50>}}
\title{Informatics basics}
\hypersetup{
 pdfauthor={Starovoytov Alexandr},
 pdftitle={Informatics basics},
 pdfkeywords={},
 pdfsubject={},
 pdfcreator={Emacs 27.2 (Org mode 9.5)}, 
 pdflang={English}}
\begin{document}

\maketitle
\tableofcontents

\begin{itemize}
\item орг инфа
\begin{itemize}
\item Коновалов Александр Владимирович
\item 89175404352
\item akonovalov@bmstu.ru
\item a.v.konovalov87@mail.ru
\end{itemize}
\end{itemize}
\section{1 лекция Основные понятия информатики}
\label{sec:orge285fe4}
\subsection{основные понятия информатики}
\label{sec:org3cc0851}
\subsubsection{Опр данных}
\label{sec:org47898a7}
данные это представление фактов, понятий инструкций в форме, приемлемой
для обмена, интерпретации или обработки человеком или с помощью
автоматических средств
\subsubsection{опр алгоритма}
\label{sec:org5d60aeb}
конечная совокупность точно заданных правил решения прозвольного класса
задач или набор инструкций описывающий порядок действий исполнителя для
решения некотрой задачи
\subsubsection{св-ва алгоритма}
\label{sec:orgc0eb683}
\begin{enumerate}
\item дискретность
делится на отдельные эл-е части отвечающие за определенные действия
\item детерминированность
на одних и тех же входных данных - один и тот же результат
\item понятность
эл-ты алгоритма должны быть понятны исполнителю
\item завершаемость
не завершаются - вычислительный процесс
\item массовость
на многих входных данных
\item результативность
\end{enumerate}

\subsubsection{опр компьютерной программы}
\label{sec:org7ebdcf1}
это алгоритм записанный на некотором языке программирования
\subsubsection{язык программирования}
\label{sec:org8537d29}
формальный язык предназначенный для записи компьютерных программ
\subsubsection{компьютер}
\label{sec:org599e00a}
программно управляемое устр-во для обработки информации
\subsubsection{подпрограмма}
\label{sec:orgf2e6352}
подпрограмма - некоторый именованный блок кода \\
вызывающая программа приостанавливается, управление передается
подпрограмме \\
по завершению управление передается обратно
\subsubsection{сопрограмма}
\label{sec:org4b86166}
в отличие от подпрограммы работает поочередно с вызывающей программой \\
при вызове возобновит выполнение с момента где остановилась
\subsection{парадигмы программирования}
\label{sec:org1b1453c}
совокупность идей и понятий определяющих стиль написания компьютерных
программ (подход к программированию) это способ концептуализации
определяющий организацию вычисл и структурирование работы выполняемой
компьютером
\subsubsection{основные группы парадигм}
\label{sec:org4d1acdf}
\begin{enumerate}
\item императивные
способ записи программ в котором указывается последовательность
действий \\
основной признак - оператор присваивания (меняющий значение переменной)
\begin{enumerate}
\item структурное программирование
программа является композицией блоков с одним входом и
одним выходом \\
есть операторы ветвления и цикла, но нет goto
\item процедурное программирование
совокупность подпрограмм где одни подпрограммы вызывают другие \\
подпрограмма - некоторый именованный блок кода \\
вызывающая программа приостанавливается, управление передается
подпрограмме \\
по завершению управление передается обратно
\item объектно-ориентированное
программа рассматривается как набор некоторых взаимодействующих
объектов \\
объект сочетает в себе данные и методы их обработки
методы вызываются в ответ на сообщение \\
посылаем сообщение объекту -> вызывает метод -> возвращает ответ \\
объекты объединяются в классы
\end{enumerate}
\item декларативные
способ записи программ в котором описывается взаимосвязь между
данными, описывается цель, но не последовательность шагов ее вычисления
\begin{enumerate}
\item функциональное
алгоритм записывается как набор взаимосвязанных функций, функции
рассматриваются с математической точки зрения \\
описывает взаимосвязь между данными и результатом
\begin{verbatim}
      x = f(y) + g(z);
\end{verbatim}
\item логическое (Prolog + отчасти SQL) \\
алгоритм описывает взаимосвязь между понятиями \\
выполнение программы сводится к выполнению запросов
\end{enumerate}
\item метапрограммирование
программа рассматривается как данные/объект для другой программы
\begin{enumerate}
\item программу пишут программы
перед выполнением транслируется \\
например: макросы, генераторы кода, шаблоны С++(фигня нечитаемая зачем он постоянно в руках часы крутит попит симпл димпл круче нет попит симпл димпл попит симпл димпл маленький красивый попит большой и милый)
\item программа взаимодействует с вычислительной средой
рефлексия или интроспекция - самоанализ \\
например посмотреть поля и значения в объекте класса
\end{enumerate}
\end{enumerate}

\subsubsection{примеры}
\label{sec:org0774926}
\begin{verbatim}
ИП: sort(array);
ФП: sorted_list = sort(list)
    sorted_list - константа, т.к. нет присваивания
ЛП: sort(unsorted, sorted)
\end{verbatim}

\subsection{язык Scheme}
\label{sec:org22ab5fe}
\subsubsection{информация о языке}
\label{sec:orga938ddd}
Lisp 1950-е годы Джон МакКарти \\
LISt Processing \\
Scheme 1970-е годы Абельсон и Сассман \\
изучаем R5RS \\
современная редакция R7RS
\begin{itemize}
\item другие языки семейства LISP:
\begin{itemize}
\item Common Lisp
\item Clojure
\item Racket
\end{itemize}
\end{itemize}
\end{document}
